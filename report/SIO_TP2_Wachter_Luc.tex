% Preamble
% ---
\documentclass[paper=a4, fontsize=13pt]{scrartcl}

% Packages =========================================================================
% ---
\usepackage{geometry}
\geometry{
  a4paper,
  total={170mm,257mm},
  left=20mm,
  right=20mm,
  top=30mm,
  bottom=35mm,
}
\usepackage[utf8]{inputenc}   % UTF-8 support
\usepackage[french]{babel}
\usepackage{amsmath,amsfonts} % Advanced math typesetting
\usepackage{listings}         % Source code formatting and highlighting
\usepackage{lastpage}         % Reference to last page
\usepackage[parfill]{parskip} % No indent at start of new line
\usepackage{color}
  \definecolor{codegreen}{rgb}{0,0.6,0}
  \definecolor{codegray}{rgb}{0.5,0.5,0.5}
  \definecolor{backcolour}{rgb}{0.95,0.95,0.92}
\usepackage{fancyhdr}         % Create headers and footers
	\pagestyle{fancy}

% Create horizontal rule command with 1 argument of height
\newcommand{\horrule}[1]{\rule{\linewidth}{#1}}

% Title page things ================================================================
\title{
  \normalfont \normalsize
  \textsc{HEIG-VD - Simulation et Optimisation} \\[10pt]
  \horrule{1pt} \\[0.4cm] % Thin top horizontal rule
  \huge Travail Pratique 2~:\\Méthode de Monte-Carlo \\
  \horrule{2pt} \\[0.4cm] % Thick bottom horizontal rule
}
\author{Luc \bsc{Wachter}}
\date{28 mai 2019}

% Header and footer things =========================================================
\renewcommand{\footrulewidth}{0.4pt}
\lhead{SIO - Travail pratique 2}
\rhead{Luc Wachter}
\lfoot{28.05.2019}
\cfoot{}
\rfoot{Page \thepage \ sur \pageref{LastPage}}

% Style for code listings ==========================================================
\lstdefinestyle{codestyle}{
    backgroundcolor=\color{backcolour},
    commentstyle=\color{codegreen},
    numberstyle=\tiny\color{codegray},
    basicstyle=\footnotesize,
    breakatwhitespace=true,
    breaklines=true,
    captionpos=b,
    keepspaces=true,
    numbers=left,
    numbersep=5pt,
    showspaces=false,
    showstringspaces=false,
    showtabs=false,
    tabsize=2
}
\lstset{style=codestyle}

% Document =========================================================================
% ---
\begin{document}

% Set language for listings
\lstset{language=Java}

\maketitle

% Début Section 1 ==================================================================
\section{Introduction}
De quoi s'agit-il ? C'est pourtant clair.
\\
Si vous ne le savez pas, c'est que vous n'êtes pas prêts.

\newpage

% Début Section 2 ==================================================================
\section{Calcul de N}
Nous connaissons le calcul pour la largeur \(\Delta_{I_c}\) de l'intervalle de confiance, et pouvons en déduire celui pour sa demi-largeur sans efforts~:
\begin{align*}
  \Delta_{I_c} = 2\cdot z_{1-\frac{\alpha}{2}}\cdot \dfrac{s}{\sqrt{n}} \\
  \Rightarrow \dfrac{\Delta_{I_c}}{2} = z_{1-\frac{\alpha}{2}}\cdot \dfrac{s}{\sqrt{n}}
\end{align*}

% Début Section 3 ==================================================================
\section{Choix d'implémentation}

\subsection{Implémentation d'une expérience}
Implémentation d'\texttt{Experiment}

\begin{lstlisting}[frame=single]
public double execute(Random rnd) {
    // Generate two points in the unit square and return the distance
    // between them
    return Point2D.distance(rnd.nextDouble(), rnd.nextDouble(),
                            rnd.nextDouble(), rnd.nextDouble());
}
\end{lstlisting}

\subsection{Méthode principale de simulation}
Méthode \texttt{simulateTillGivenCIHalfWidth}

\subsection{Code client}
La méthode main, quoi, avec les tests.

% Début Section 4 ==================================================================
\section{Résultats}
Insérer graphiques super cool ici.

% Début Section 5 ==================================================================
\section{Conclusion}
C'était \textbf{trivial}.

\end{document}
