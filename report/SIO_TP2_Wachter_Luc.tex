% Preamble
% ---
\documentclass[paper=a4, fontsize=11pt]{scrartcl}

% Packages =========================================================================
% ---
\usepackage{geometry}
\geometry{
  a4paper,
  total={170mm,257mm},
  left=20mm,
  right=20mm,
  top=30mm,
  bottom=35mm,
}
\usepackage[utf8]{inputenc}   % UTF-8 support
\usepackage[french]{babel}
\usepackage{amsmath,amsfonts} % Advanced math typesetting
\usepackage{listings}         % Source code formatting and highlighting
\usepackage{lastpage}         % Reference to last page
\usepackage[parfill]{parskip} % No indent at start of new line
\usepackage{color}
  \definecolor{codegreen}{rgb}{0,0.6,0}
  \definecolor{codegray}{rgb}{0.5,0.5,0.5}
  \definecolor{backcolour}{rgb}{0.95,0.95,0.92}
\usepackage{fancyhdr}         % Create headers and footers
	\pagestyle{fancy}

% Create horizontal rule command with 1 argument of height
\newcommand{\horrule}[1]{\rule{\linewidth}{#1}}

% Title page things ================================================================
\title{
  \normalfont \normalsize
  \textsc{HEIG-VD - Simulation et Optimisation} \\[10pt]
  \horrule{1pt} \\[0.4cm] % Thin top horizontal rule
  \huge Travail Pratique 2~:\\Méthode de Monte-Carlo \\
  \horrule{2pt} \\[0.4cm] % Thick bottom horizontal rule
}
\author{Luc \bsc{Wachter}}
\date{28 mai 2019}

% Header and footer things =========================================================
\renewcommand{\footrulewidth}{0.4pt}
\lhead{SIO - Travail pratique 2}
\rhead{Luc Wachter}
\lfoot{28.05.2019}
\cfoot{}
\rfoot{Page \thepage \ sur \pageref{LastPage}}

% Style for code listings ==========================================================
\lstdefinestyle{codestyle}{
    backgroundcolor=\color{backcolour},
    commentstyle=\color{codegreen},
    numberstyle=\tiny\color{codegray},
    basicstyle=\footnotesize,
    breakatwhitespace=true,
    breaklines=true,
    captionpos=b,
    keepspaces=true,
    numbers=left,
    numbersep=5pt,
    showspaces=false,
    showstringspaces=false,
    showtabs=false,
    tabsize=2
}
\lstset{style=codestyle}

% Document =========================================================================
% ---
\begin{document}

% Set language for code listings
\lstset{language=Java}

\maketitle

% Début Section ==================================================================
\section{Introduction}

Dans ce travail pratique, nous nous intéressons au calcul de la distance moyenne entre deux points du carré unité. Il s'agit donc de calculer l'espérance de la distance euclidienne séparant deux points indépendants et identiquement distribués dans le carré unité $[0;1] \times [0;1]$.

Si on note $P(x_1;y_1)$ et $Q(x_2;y_2)$ deux points du carré unité et $D$ la distance euclidienne qui les sépare, l'espérance de cette distance est
\begin{equation*}
  \text{E}(D) = \int _0^1\int _0^1\int _0^1\int _0^1\sqrt{\:\left(x_2-x_1\right)^2+\left(y_2-y_1\right)^2}dy_2dy_1dx_2dx_1.
\end{equation*}
C'est cette espérance que l'on veut calculer.

Pour ce faire, nous allons réaliser une simulation de Monte-Carlo afin de générer des estimateurs $\widehat{D}$ de $\text{E}(D)$ à partir d'échantillons de populations générées aléatoirement.

Ce document a pour but de documenter les choix faits durant la conception du programme, mais surtout de présenter les résultats obtenus, de les interpréter et de les discuter.

% Début Section ==================================================================
\section{Approche utilisée}

Je fais ça.

% Début Section ==================================================================
\section{Correspondance des symboles}

Les symboles mathématiques proposés dans la donnée du travail ne sont bien sûr pas propices à une utilisation en Java. Le tableau \ref{table:correspondance} permet donc de faire le lien entre les symboles utilisés dans ce document et ceux utilisés dans le programme.

\begin{table}
  \centering
  \begin{tabular}{|l|l|}
    \hline
    Symbole mathématique & Symbole dans le code \\
    \hline
    $N_{init}$ & initialNumberOfRuns \\
    $N_{add}$ & additionalNumberOfRuns \\
    $N$ & estimatedNumberOfRunsNeeded \\
    $\Delta_{max}$ & maxHalfWidth \\
    $s$ & standardDeviation \\
    \hline
  \end{tabular}
  \caption{Correspondance des symboles}
  \label{table:correspondance}
\end{table}

% Début Section ==================================================================
\section{Choix du nombre d'expériences à effectuer}

Après avoir lancé la simulation pour exécuter $N_{init}$ expériences, il nous faut estimer le nombre d'expériences supplémentaires nécessaires pour arriver à un intervalle de confiance à 95\% dont la demi-largeur ne dépasse pas $\Delta_{max}$.

Ce nombre d'expériences $n$ peut être estimé en utilisant l'équation pour la largeur de l'intervalle de confiance~:
\begin{equation*}
  \Delta_{I_c} = 2\cdot z_{1-\frac{\alpha}{2}}\cdot \dfrac{s}{\sqrt{n}}
\end{equation*}

Adaptons cette équation pour notre cas pratique en remplaçant la largeur de l'intervalle de confiance $\Delta_{Ic}$ par la demi-largeur maximale $\Delta_{max}$ (valeur fixée par le code appelant). Ce changement nous permet de faire fi du facteur deux du côté droit de l'équation, puisque nous nous intéressons à la demi-largeur et non à la largeur complète.
\begin{equation*}
  \Delta_{max} = z_{1-\frac{\alpha}{2}}\cdot \dfrac{s}{\sqrt{n}}
\end{equation*}

Il nous suffit alors d'isoler $n$ pour déterminer la formule à utiliser dans notre programme.
\begin{align*}
  \Delta_{max} &= z_{1-\frac{\alpha}{2}}\cdot \dfrac{s}{\sqrt{n}} \\ \\
  \Rightarrow \ \sqrt{n} &= \dfrac{z_{1-\frac{\alpha}{2}}\cdot s}{\Delta_{I_c}} \\ \\
  \Rightarrow \ n &= \dfrac{z_{1-\frac{\alpha}{2}}^2\cdot s^2}{\Delta_{I_c}^2}
\end{align*}

En utilisant $1.959964$ comme quantile de la loi normale et avec l'estimateur $s$ de l'écart-type calculé empiriquement à l'aide de nos expériences initiales, il nous est donc possible d'estimer un nombre d'expériences nécessaires pour arriver à la précision voulue.

% Début Section ==================================================================
\section{Résultats}

\subsection{Distribution des estimateurs ponctuels $\widehat{D}$}

\subsection{Couverture empirique des intervalles de confiance}

Afin d'analyser la couverture des intervalles de confiance calculés, la valeur exacte de $\text{E}(D)$ nous est donnée.
\begin{align*}
  \text{E}(D) &= \int _0^1\int _0^1\int _0^1\int _0^1\sqrt{\:\left(x_2-x_1\right)^2+\left(y_2-y_1\right)^2}dy_2dy_1dx_2dx_1 \\ \\
  &= \dfrac{2 + \sqrt{2} + 5\ln{\sqrt{2} + 1}}{15} \\ \\
  &\approx 0,5214054
\end{align*}

\textit{Insérer ici graphique de fou de la slide 52}

% Début Section ==================================================================
\section{Conclusion}

C'était \textbf{trivial}.

\end{document}
